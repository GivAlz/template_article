\section{Introduction}

This is a template. You're free to use it or copy it as you wish.
There is a .gitignore file with common latex outputs and .ps/.pdf files:
if you want to include these modify the .gitignore.\\

Moreover here are some tips which, in my experience, few scientists know:
if you're a (PhD) student or a professor who doesn't know what "git" is 
make sure to read the following silly text.\\

Using the command \verb=\input{filename}= (without the .tex ending) is
like writing in the original file.\\

Like the command contained in the next page it is very useful to organize
work in various cases:
\begin{itemize}
\item By commenting all sections but one compiling time can be greatly
reduced: this is useful if you are working on a formula, positioning an image
or, in general, trying to get the output as you want in one particular 
section of the text
\item It helps with bigger projects also by allowing better git usage.
\end{itemize}

If you don't know what git is then start using it: it allows for a backup
of all the versions of your article. Moreover, splitting wisely the text,
it makes collaborative work extremely easy even for text editing.\\

Git is meant for programmers but can really help even in writing documents:
start using it now!

To know about it try \url{https://git-scm.com/} and/or google it.
There are many websites for keeping your git stuff: if you wish a private
repository (for instance if you're writing an article) just know that some
of these websites offer you free private repositories, with some limitations
which shouldn't be a problem.
